\documentclass[a4paper,12pt]{article}

\usepackage{authblk}
\usepackage[utf8]{inputenc}
\usepackage[spanish]{babel}

\topmargin= 0cm
\evensidemargin= -1cm
\oddsidemargin= -1cm
\textheight= 23cm
\textwidth= 18cm

\title{\vspace{-3.5cm}Introducción de Github}
\author{\vspace{-6mm}Santiago Hernández}
\date{\vspace{-5mm}Julio 2021}

\begin{document}
  \maketitle
  Git Hub es una herramineta que nos ayuda a editar y mantener actulaizados archivos que estemos trabjando con ellos. También es una herramienta que nos ayuda poder trabajar con otras personas y tener un trabajo colaborativo.

  Los principales comandos para poder crear tu repositorio y poder usarlo son:

 \begin{itemize}
   \item \textit{git init:} Cuando estamos creando un nuevo repositorio en nuestra computadora, es necesario este comando para poder iniciar un repositorio en la raiz del archivo tendremos que correr este comando.
   \item \textit{git status:} Este comando nos ayuda para que git sepa que archivos existen.
   \item \textit{commit:} Es un cambio indiviudal al archivo.
 \end{itemize}

\end{document}
